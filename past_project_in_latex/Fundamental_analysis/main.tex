\documentclass{article}
\usepackage[utf8]{inputenc}
\usepackage{graphicx}
\usepackage{hyperref}
\usepackage[margin=1in]{geometry} 
\title{How I develop directional bias from fundamental and sentiment analysis in currency trading}
\begin{document}
\maketitle

\section{My opinion toward Fundamental analysis in equity market}
I'm not Warren Buffett, so I'm not going to say something like, "Technical analysis is like looking in the rearview mirror while driving. Fundamental analysis, on the other hand, is like looking through the windshield," because I think that's a false statement. Technical analysis can also provide a forward-looking perspective too if you change the timeframe to a week or a month. To share with you, I began my investment journey with a book called "Gone Fishing with Buffett" and to be honest, I don't think it's beneficial for my trading journey as it oversimplifies so many concepts in fundamental analysis. 

\section{My opinion toward Fundamental Analysis in Forex market}
I think that the context of Fundamental Analysis in Forex and Equity is extremely different. In equity trading, everyone wants to make money from the market, so someone with better insight - usually an institutional trader - often comes out on top. Retail traders who buy a stock with strong fundamentals usually do so after the financial report has been released and the P/E ratio has already skyrocketed. While a stock with a solid financial report and a low P/E ratio may seem promising to retail traders, only those with more insider information know why the P/E ratio is low - and once again, those are usually institutional traders. My point is that making accurate fundamental analyses requires too much effort for retail traders.

However, in the Forex market, there is a group of players called "Commercial" who account for almost half of the transactions in some currency pairs. They are not there to make money from the market since it is not their core business. Their primary goal is to stabilize the value of the currency they are holding. Hence, with solid fundamental analysis, we can catch some profitable opportunities from them. 

\section{Sentiment Analysis}
Before I dive into fundamental analysis, another easier analysis that indicates the supply and demand dynamics of the market is sentiment analysis. There are two types of sentiment analysis - retail sentiment and institutional sentiment. The analysis below will explore how the sentiment of these two groups of investors affects the price.

\subsection{Retail Sentiment}
Before we dive into the data analysis part, let's see what folk on the internet think about retail sentiment. \\ \\ 
\textbf{Trader Nick and most youtubers}: 95\% of retail trader is on the wrong side of the market, hence, we should take the opposite position of the retail trader. \\ \\
\textbf{FXEmpire}: Forex is the market where the smartest individual traders see where the “dumb money” goes, so it's best to avoid trading along with it or against it. Because you can never know how the smartest money would exploit the opportunity. \\ \\
\textbf{reddit people}: I lost a lot of money by trying to short at "resistances" and buy at "supports so as retail traders. \\ \\
\textbf{DailyFX}: Publish a regular contrarian index openly stating that its market view on a given pair is diametrically opposed to that of the retail sector. \\ \\
Above are the opinions of people on the internet, it seems everyone tends to agree that we should trade against retail. However, let's explore the data. \\ \\ 

\subsubsection{Validating folk's opinion with data}
\begin{center}
    \includegraphics[scale=0.5]{p1.png}    
\end{center}
The screenshot above displays the retail sentiment and price of EURUSD. It can be observed that when the retail crowd holds long positions, the price tends to decrease, whereas when retail investors have short positions, the price tends to increase. However, we need to validate this observation numerically. I have tried different approaches to obtain the data from the chart above, including a computer vision algorithm, but it has not been successful. If I were to obtain the data from a provider such as TraderMade, it would be extremely expensive and FXCM provides historical sentiment data only to premium accounts. Another way to obtain the data is to run a Python script on the cloud every hour to extract the sentiment. I will explore this option further when I have more resources. Alternatively, I could manually input the data into Excel, but this would require a significant amount of effort and it's a dump way to spend time.  \\ \\
%--------------------------------------------------------%

Even though it may seem foolish to manually collect data from a website, my curiosity overpowered me this time. I spent hours collecting position sizes from the website manually, and here is the analysis. \\

\subsubsection{Visual Inspection}
Many YouTubers and bloggers on Reddit have previously emphasized the importance of taking the opposite side of retail sentiment. However, when we plot retail sentiment against price, it tells a completely different story. Let's analyze the chart below one by one, and I found this insight extremely interesting. \\ \\ 
\textbf{1) Scaled price vs percentage of long position}
\begin{center}
    \includegraphics[scale=0.7]{p2.png}    
\end{center}
My interpretation of the figure is that even though retail sentiment is often considered to be incorrect, when the sentiment on the long side reaches its peak, the price usually goes up. There are many assumptions behind this observation, but I will discuss them in more detail later in the analysis. \\ \\ 

\textbf{2) Scaled price vs scaled short position size}
\begin{center}
    \includegraphics[scale=0.7]{p3.png}    
\end{center}
There are two observations from the figure above \\
1) When the short position size is low, the price usually goes up (4 out of 5 times). \\
2) When the short position size is at its peak, the price might not go down because it goes against the main trend.  \\ \\

\textbf{3) Scaled price vs scaled long position size}
\begin{center}
    \includegraphics[scale=0.7]{p4.png}    
\end{center}
From the figure above, it can be observed that when the long position reaches its peak, the price tends to go up from there.\\ \\

From the above three charts, it is difficult to conclude that retail sentiment is a contrarian indicator, as we often see the price go up when most retail traders are in a long position, and the price go down when most of them are in a short position. However, it can be concluded that retail traders usually take profit too soon. \\ 

one potential strategy to take a long position is to wait for retail traders to take an extreme long position when the price dip, and then entering the trade when they are taking profit.

\subsubsection{Correlation Coefficient}
The next step is to find the correlation coefficient between price and retail sentiment. Since I have the following columns in my dataframe [close, Short, Long, Short\_percentage, Long\_percentage, Short\_scaled, Long\_scaled, close\_scaled], , I should end up with 12 correlation coefficients. The results of the correlation coefficients are as follows:
\begin{center}
    \includegraphics[scale=0.7]{p5.png}    
\end{center}
The result is not surprising; it seems that 'Long\_percentage' and 'Long\_scaled' are highly correlated with the price. I assume it's because EURUSD is on an increasing trend. However, it's worth noting that 'Short\_percentage' and 'Short\_scaled' also show correlation even if it's not in decreasing trend.

\subsubsection{Granger causality}
What is Granger causality? It seems there is a guy who won Nobel Prize from it. I, myself, with a BSc in Statistics and an M.Eng. in ORIE, as well as having passed numerous actuarial exams, have no idea what the hell it is. However, it seems that people have been using it to analyze "whether one time series can predict another time series." I will walk through the mathematics behind it just once, to know what I'm doing.
\\ \\ Say we have a time series y, \\
\begin{center}
$Y_t = \alpha_0 + \alpha_1 Y_{t-1} + \alpha_2 Y_{t-2} + \cdots + \alpha_p Y_{t-p} + \varepsilon_t$
\end{center}
Next, we model Y using both its own past values and the past values of X
\begin{center}
$Y_t = \beta_0 + \beta_1 Y_{t-1} + \beta_2 Y_{t-2} + \cdots + \beta_p Y_{t-p} + \gamma_1 X_{t-1} + \gamma_2 X_{t-2} + \cdots + \gamma_p X_{t-p} + \eta_t$
\end{center}
Then, we compare the residuals from the two models ($\varepsilon_t$ and $\eta_t$) to determine if including the past values of X significantly improves the prediction of Y. This can be done using an F-test, which tests the null hypothesis that the coefficients of the X variable ($\gamma_1$,$\gamma_2$,...) are all equal to zero. If the F-test rejects the null hypothesis, then we can conclude that the past values of X help predict the future values of Y
\\ \\
Next step, I'll perform Granger causality between the price and Retail Sentiment.
\begin{center}
    \includegraphics[scale=0.7]{p6.png}    
\end{center}
\begin{center}
    \includegraphics[scale=0.7]{p7.png}    
\end{center}
Based on the results, it can be concluded that there is no significant Granger causality relationship between retail sentiment and price. \\ 
\subsubsection{Next Step on explore retail sentiment}
Although the Granger causality relationship may not hold, the correlation between the retail sentiment graph and price still provides valuable insights, particularly when utilized with the right directional bias. Further exploration of this relationship across various timeframes, such as the 1-hour timeframe, and with different assets such as XAUUSD and USDJPY, should be explored. 

\subsection{Institutional Money Sentiment}
Many YouTubers claim that institutional money is referred to as the "smart money." In this section, I aim to prove this using the COT report. If you are unfamiliar with the COT report, google or gpt it and come back.

\subsubsection{Visual Inspection}
In the following section, I want to see the price with, long/short position, and long/short percentage, with all values normalized between 0 and 1. By doing so, I intend to analyze their respective behaviors and interactions. \\

\textbf{1) Scaled price vs scaled long x long percentage}
\begin{center}
    \includegraphics[scale=0.7]{p8.png}    
\end{center}

Upon examining the graph comparing normalized closing prices and normalized long positions, we can observe a correlation and predictive capability in the Commitment of Traders (CoT) data. For instance, around index 50, institutional money considerably reduces long positions followed by a consistent decline in price. This pattern is also visible in the percentage graph, although it may be less clear due to its potentially misleading nature. When long positions decrease significantly, short positions decrease as well, which can make the long percentage appear to be less affected. Therefore, I conclude that the normalized long positions provide more valuable insights than the long percentage. I will continue to investigate the graph focusing on short positions to gain further understanding. \\

\textbf{2) Scaled price vs scaled short x short percentage}
\begin{center}
    \includegraphics[scale=0.7]{p9.png}    
\end{center}
We can observe an inverse relationship between price and short position, with two noteworthy observations: \\
1) At index 50, when the long position significantly decreases, the short position continues to increase and reaches its peak at index 150. It appears as though the long positions were converted to short positions. Interestingly, even when the short position reaches its peak and begins to decrease, the price continues to decline for an extended period. \\ 
2) Similar to the observation regarding the long percentage, even when the short position increases substantially around index 100, the short percentage remains relatively stable. This leads to the conclusion that we should prioritize the normalized values over the percentage-based representations for a clearer understanding of the data.

\subsubsection{Correlation Coefficient}
Like retail sentiment analysis, the next step is to find the correlation coefficient between price and institutional position. The results of the correlation coefficients are as follows:
\begin{center}
    \includegraphics[scale=1]{p10.png}    
\end{center}
As anticipated, the scaled long positions exhibit a stronger correlation with the price compared to the long percentage. Additionally, long positions demonstrate a positive correlation with the price, while short positions reveal a negative correlation. However, it is important to note that the magnitude of these correlations is not particularly large.
\subsubsection{Granger causality}

\begin{center}
    \includegraphics[scale=1]{p11.png}    
\end{center}

\begin{center}
    \includegraphics[scale=1]{p12.png}    
\end{center}

There is no significant Granger causality relationship between institutional money and price.


\subsection{Reflection and next step on sentiment analysis}
When attempting to establish a mathematical correlation between sentiment data and price, the results are often not very significant. Nonetheless, visual examination of the data appears to provide greater insight. For instance, by using retail sentiment as an oscillator and institutional money sentiment to develop a directional bias, more accurate results can be obtained.\\ \\ 
For the subsequent phase, it is recommended to carry out the same analysis for each currency. However, gathering retail sentiment data for each currency can be a time-consuming task. As a solution, I plan to write a Python program to automatically scrape retail sentiment data on an hourly basis going forward.


\end{document}
\begin{center}
    \includegraphics[scale=0.7]{p1.png}    
\end{center}